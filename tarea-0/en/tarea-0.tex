\documentclass[spanish, fleqn]{article}
\usepackage{babel}
\usepackage[utf8]{inputenc}
\usepackage{amsmath}
\usepackage{amsfonts}
\usepackage[colorlinks, urlcolor=blue]{hyperref}
\usepackage{fourier}
\usepackage[top = 2.5cm, bottom = 2cm, left = 2cm, right = 2cm]{geometry}

\newcommand{\num}{0}

\title{Informática Teórica\\
       Tarea \#\num \\
       ``¡Precalentamiento!''}
\author{Language Warriors}
\date{4 de septiembre de 2017}

\begin{document}
\maketitle
\thispagestyle{empty}

\section*{Preguntas}

  Analice los patrones de búsqueda que ofrece el programa \texttt{grep(1)}
  propio de Unix.
  Usando éstas,
  escriba
  (¡y pruebe!)
  patrones para las siguientes.
  Note que \texttt{grep(1)} busca el patrón dado en la línea,
  y retorna la línea si el patrón calza en alguna parte.
  Aproveche las abreviaturas para escribir expresiones
  lo más cortas posibles:
  \begin{enumerate}
  \item
    Palabras que tienen una 'a', entre \(3\) y \(7\) 'b' seguidas por 'c'
  \item
    Líneas que comienzan con una letra mayúscula
  \item
    Líneas que terminan en '.',
    posiblemente seguidas por espacios
  \item
    Contiene 'a' seguido por una o varias 'b' o 'c',
    terminando en 'd'
  \item
    Contiene 'a' seguido por una o varias veces 'ab', terminando en 'd'
  \end{enumerate}
  Debe entregar para cada una de las anteriores el patrón
  como se escribe en la línea de comandos
  (en \LaTeX{} puede serle útil el comando \verb!\verb+...+!
   para escribir texto con caracteres que tienen significado especial),
  junto con una \emph{breve} explicación.

\input{condiciones}
  \vfill\hfill LW/HvB/\LaTeXe
\end{document}

%%% Local Variables:
%%% mode: latex
%%% TeX-master: t
%%% End:

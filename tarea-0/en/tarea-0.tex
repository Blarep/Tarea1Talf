\documentclass[spanish, fleqn]{article}
\usepackage{babel}
\usepackage[utf8]{inputenc}
\usepackage{amsmath}
\usepackage{amsfonts}
\usepackage[colorlinks, urlcolor=blue]{hyperref}
\usepackage{fourier}
\usepackage[top = 2.5cm, bottom = 2cm, left = 2cm, right = 2cm]{geometry}

\newcommand{\num}{0}

\title{Informática Teórica\\
       Tarea \#\num \\
       ``¡Precalentamiento!''}
\author{Andrés Navarro \\ 201673001-K}
\date{4 de septiembre de 2017}

\begin{document}
\maketitle
\thispagestyle{empty}

\section*{Preguntas}

  Analice los patrones de búsqueda que ofrece el programa \texttt{grep(1)}
  propio de Unix.
  Usando éstas,
  escriba
  (¡y pruebe!)
  patrones para las siguientes.
  Note que \texttt{grep(1)} busca el patrón dado en la línea,
  y retorna la línea si el patrón calza en alguna parte.
  Aproveche las abreviaturas para escribir expresiones
  lo más cortas posibles:
  \begin{enumerate}
  \item
    Palabras que tienen una 'a', entre \(3\) y \(7\) 'b' seguidas por 'c'
  	\begin{verbatim}
    grep -E '[^a]*a[^a]*b{3,7}c[^a]*|[^a]*b{3,7}c[^a]*a[^a]*'
    \end{verbatim}
    En primer lugar '[\verb+^+a]' significa que se hará un calce con cualquier carácter menos 'a'. Ahora bien, tenemos dos opciones, la primera es que existan cero o más caracteres que no son una 'a'([\verb+^+a]), luego una 'a', cero o más caracteres que no son una 'a', entre 3 y 7 'b' (b\{3-7\}), luego una 'c' y finalmente cero o más caracteres que no son una 'a'. En la segunda opción lo que ocurre es que aparecen las  letras 'b' junto con la 'c' antes de la letra 'a'.
  
  \item
    Líneas que comienzan con una letra mayúscula
    \begin{verbatim}
    grep '^[A-Z]'
    \end{verbatim}
    Para que comiencen con una letra mayúscula basta con poner un \verb+^+ antes de nuestro conjunto para declarar que buscamos cualquier letra mayúscula entre la A y la Z al inicio de la línea.
    
  \item
    Líneas que terminan en '.',
    posiblemente seguidas por espacios
    \begin{verbatim}
    grep '\.[ ]*$'
    \end{verbatim}    
    Primero se escribe lo que queremos poner al final de la línea, como queremos un punto debemos ocupar '\verb+\+' , ya que el punto es un metacáracter, luego, como pueden aparecer cero o más espacios ocupamos  '*'  para declararlo, finalmente usamos '\$' para declarar que lo anterior aparezca al final de la línea.
  
  \item
    Contiene 'a' seguido por una o varias 'b' o 'c',
    terminando en 'd'
    \begin{verbatim}
    grep -E '.*a[b|c]+.*d\$'
    \end{verbatim}
    Pueden haber cero o más caracteres antes de la 'a', por lo que usamos '.*', luego de esta usamos '[b|c]' para elegir entre ambas letras seguido de un '+', ya que este indica que lo anterior aparecerá una o más veces. Luego pueden venir más caracteres, por lo que empleamos '.*' y finalmente una 'd' seguida de un '\$' para declarar que es lo último que debe aparecer.
    
    Usamos '-E' para declarar que es una expresión regular, ya que grep por si solo no incluye algunas funciones (como '+').
  \item
    Contiene 'a' seguido por una o varias veces 'ab', terminando en 'd'
    \begin{verbatim}
    grep -E '.*a(ab)+.*d\$'
    \end{verbatim}    
    Primero puede haber varios caracteres, por lo que usamos '.*', después aparecerá la 'a', seguida de esta declaramos que debe aparecer '(ab)' una o más veces con '+', luego pueden haber más caracteres por lo que empleamos '.*'. Finalmente escribimos 'd\$' indicando que esta es la última letra que queremos que aparezca.
    Usamos '-E' para declarar que es una expresión regular y así ocupar elementos como '()' y '+'.
  \end{enumerate}
  
  
  Debe entregar para cada una de las anteriores el patrón
  como se escribe en la línea de comandos
  (en \LaTeX{} puede serle útil el comando \verb!\verb+...+!
   para escribir texto con caracteres que tienen significado especial),
  junto con una \emph{breve} explicación.

\end{document}

%%% Local Variables:
%%% mode: latex
%%% TeX-master: t
%%% End:

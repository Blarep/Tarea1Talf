\documentclass[spanish, fleqn]{article}
\usepackage{babel}
\usepackage[utf8]{inputenc}
\usepackage{amsmath}
\usepackage{amsfonts}
\usepackage[colorlinks, urlcolor=blue]{hyperref}
\usepackage{fourier}
\usepackage[top = 2.5cm, bottom = 2cm, left = 2cm, right = 2cm]{geometry}

\newcommand{\num}{0}

\title{Informática Teórica\\
       Tarea \#\num \\
       ``¡Precalentamiento!''}
\author{Andrés Navarro \\ 201673001-K}
\date{4 de septiembre de 2017}

\begin{document}
\maketitle
\thispagestyle{empty}

\section*{Preguntas}

  Analice los patrones de búsqueda que ofrece el programa \texttt{grep(1)}
  propio de Unix.
  Usando éstas,
  escriba
  (¡y pruebe!)
  patrones para las siguientes.
  Note que \texttt{grep(1)} busca el patrón dado en la línea,
  y retorna la línea si el patrón calza en alguna parte.
  Aproveche las abreviaturas para escribir expresiones
  lo más cortas posibles:
  \begin{enumerate}
  \item
    Palabras que tienen una 'a', entre \(3\) y \(7\) 'b' seguidas por 'c'
  \item
    Líneas que comienzan con una letra mayúscula
    
    \verb+grep '^[A-Z]'+
    
    Para que comiencen con una letra mayúscula basta con poner un \verb+^+ antes de nuestro conjunto para declarar que buscamos cualquier letra mayúscula entre la A y la Z al inicio de la línea.
    
  \item
    Líneas que terminan en '.',
    posiblemente seguidas por espacios
    
    \verb+grep '\.[ ]*$'+
    
    Primero se escribe lo que queremos poner al final de la línea, como queremos un punto debemos ocupar '\verb+\+'  ya que el punto es un meta cáracter, luego, como pueden aparecer cero o más espacios ocupamos  '*'  para declarar que puede aparecer cero o más veces, finalmente usamos '\$' para declarar que lo anterior queremos que aparezca al final de la línea.
  
  \item
    Contiene 'a' seguido por una o varias 'b' o 'c',
    terminando en 'd'
    
    \text{ grep -E '.*a[b|c]+.*d\$'}
    
    Pueden haber cero o más letras antes de la 'a' por lo que usamos '.*', luego de esta usamos [b|c] para elegir entre ambas letras seguido de un '+' ya que este indica que lo anterior aparecerá una o más veces. Luego pueden venir más letras '.*' y finalmente una 'd' seguida de un '\$' para declarar que es lo último que debe aparecer.
    
    Usamos '-E' para declarar que es una expresión regular, ya que grep por si solo no incluye algunas funciones (como '+').
  \item
    Contiene 'a' seguido por una o varias veces 'ab', terminando en 'd'
    
    \text{grep -E '.*a(ab)+.*d\$'}
    
    Primero pueden haber varios caracteres '.*', después aparecerá la 'a', seguida de esta declaramos que debe aparecer (ab) una o más veces con '+', luego pueden haber más caracteres por lo que escribimos '.*'. Finalmente escribimos 'd\$' indicando que esta es la última letra que queremos que aparezca.
    Usamos '-E' para declarar que es una expresión regular para ocupar elementos como '()' y '+'.
  \end{enumerate}
  
  
  Debe entregar para cada una de las anteriores el patrón
  como se escribe en la línea de comandos
  (en \LaTeX{} puede serle útil el comando \verb!\verb+...+!
   para escribir texto con caracteres que tienen significado especial),
  junto con una \emph{breve} explicación.

\end{document}

%%% Local Variables:
%%% mode: latex
%%% TeX-master: t
%%% End:
